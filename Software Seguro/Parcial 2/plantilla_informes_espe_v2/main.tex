% main.tex - Plantilla académica APA en LaTeX
% --------------------------------------------------------------------
% © DESARROLLADO POR ANGEL CUDCO - PLANTILLA DE LIBRE USO
% 
% CARACTERÍSTICAS:
% - Formato APA 7ma edición
% - Numeración automática de tablas/figuras
% - Bibliografía con biblatex-apa
% - Configuración en español
% 
% LICENCIA: Libre uso, modificación y distribución
% AUTOR: Angel Cudco
% VERSIÓN: 1.0 | 2025
% --------------------------------------------------------------------

\documentclass[12pt, a4paper, spanish]{report}

% Codificación e idioma
\usepackage[utf8]{inputenc}
\usepackage[T1]{fontenc}
\usepackage[spanish]{babel}
\usepackage{csquotes}

% Márgenes
\usepackage{geometry}
\geometry{margin=2.5cm}

% Interlineado doble
\usepackage{setspace}
\doublespacing

% Sangría y justificación
\setlength{\parindent}{1.27cm}
\usepackage{ragged2e}
\justifying

% Tablas, figuras y matemáticas
\usepackage{graphicx}
\usepackage{booktabs}
\usepackage{float}
\usepackage{amsmath}

% Numeración continua de figuras y tablas (no por capítulo)
\usepackage{chngcntr}
\counterwithout{figure}{chapter}
\counterwithout{table}{chapter}

% Formato de captions y figuras estilo APA (número arriba, título en cursiva debajo)
\usepackage{caption}
\usepackage[flushmargin]{footmisc}

\captionsetup[table]{
  labelfont=bf,
  textfont=it,
  labelsep=space,
  name=Tabla,
  justification=centering,
  singlelinecheck=false,
  position=top
}

\captionsetup[figure]{
  labelfont=bf,
  textfont=it,
  labelsep=space,
  name=Figura,
  justification=centering,
  singlelinecheck=false,
  position=top
}

% Hipervínculos
\usepackage[colorlinks=true, citecolor=black, linkcolor=black]{hyperref}

% Bibliografía APA 7
\usepackage[
  style=apa,
  backend=biber,
  sortcites=true,
  natbib=true,
  uniquename=false
]{biblatex}
\DeclareLanguageMapping{spanish}{spanish-apa}
\addbibresource{referencias.bib}

% Traducción de etiquetas
\addto\captionsspanish{
  \renewcommand{\refname}{REFERENCIAS BIBLIOGRÁFICAS}
  \renewcommand{\tablename}{Tabla}
  \renewcommand{\listtablename}{Listado de Tablas}
  \renewcommand{\figurename}{Figura}
  \renewcommand{\listfigurename}{Listado de Figuras}
}

% Formato para el listado de figuras y tablas
\usepackage{tocloft}
\renewcommand{\cftfigpresnum}{Figura }
\renewcommand{\cftfigaftersnum}{}
\renewcommand{\cftfigaftersnumb}{\quad}
\setlength{\cftfignumwidth}{3em}

\renewcommand{\cfttabpresnum}{Tabla }
\renewcommand{\cfttabaftersnum}{}
\renewcommand{\cfttabaftersnumb}{\quad}
\setlength{\cfttabnumwidth}{3em}

% Índice incluye bibliografía, pero no el índice mismo
\usepackage[nottoc]{tocbibind}

% Títulos de capítulo y secciones
\usepackage{titlesec}
\titleformat{\chapter}[hang]{\normalfont\bfseries\Huge\filcenter}{}{0pt}{}
\titlespacing*{\chapter}{0pt}{0pt}{1em}

\titleformat{\section}{\normalfont\large\bfseries\raggedright}{\thesection}{1em}{}
\titlespacing*{\section}{0pt}{1em}{0.5em}

\titleformat{\subsection}{\normalfont\normalsize\bfseries\raggedright}{\thesubsection}{1em}{}
\titlespacing*{\subsection}{0pt}{0.8em}{0.3em}

% Numeración automática de figuras y tablas
\renewcommand{\thefigure}{\arabic{figure}}
\renewcommand{\thetable}{\arabic{table}}

\begin{document}

% Carátula y resumen
\begin{titlepage}

\begin{center}
\includegraphics[width=0.75\textwidth]{images/logo_espe_103.png}
\end{center}

\begin{center}
 {\Large \bf Universidad de las Fuerzas Armadas ESPE}\\[0.3cm]
{\large Departamento de Ciencias de la Computación}\\[0.3cm]
{\Large Carrera de Ingeniería en Software}\\[1.0cm]

{\Large Aplicaciones Distribuidas}\\[0.25cm]
{\large NRC: 12345}\\[0.3cm]

\large{{\bf Multiplicación de matrices usando algoritmo de Strassen}} \\[1.0cm]
 
{\bf Autores:}\\[0.5cm]
\begin{tabular}{c}
    Angel Geovanny CUDCO POMAGUALLI \\
    Estudiante 2 Estudiante 2 APELLIDO 1 APELLIDO 2 \\
    Estudiante 3 Estudiante 3 APELLIDO 1 APELLIDO 2 \\
\end{tabular}\\[0.50cm]

{\bf Docente:}\\ 
Angel Geovanny CUDCO POMAGUALLI\\[1.0cm]

{\large Sangolquí, Ecuador}\\[0.2cm]
{\large \today}

\end{center}

\end{titlepage}

\begin{abstract}
    \noindent \textbf{Objetivo:} [Breve descripción del objetivo]. \\
    \textbf{Algoritmo:} [Nombre del algoritmo]. \\
    \textbf{Métodos:} [Teórico/empírico]. \\
    \textbf{Resultados clave:} [Síntesis de 2 líneas]. \\
    \textbf{Conclusiones:} [Principales hallazgos en 2 líneas]. \\
    \vspace{12pt} % Máximo 12 líneas
\end{abstract}

% Índices
\tableofcontents
\listoftables
\listoffigures

% Cuerpo del documento
\chapter{Introducción}

\section{Problema y relevancia}
[Descripción general del problema que resuelve el algoritmo y justificación de su importancia].




\chapter{Objetivos}

\section{General}

Objetivo general 

\section{Específicos}

\begin{itemize}
    \item Objetivo específico 1
    \item Objetivo específico 2
    \item Objetivo específico 3
\end{itemize}
\chapter{Marco teórico}

El \textbf{algoritmo de Strassen} representa un hito fundamental en el ámbito de la computación matricial. Desarrollado por Volker Strassen en 1969, este algoritmo revolucionario demostró que la multiplicación de matrices podía realizarse en un tiempo inferior al $O(n^3)$ requerido por el método convencional, específicamente en $O(n^{2.81})$ \citep{Strassen1969}. Esta contribución no solo desafió la presunta optimalidad del método estándar, sino que inauguró una nueva línea de investigación en algoritmos matriciales rápidos, estableciendo un precedente para desarrollos posteriores en el campo de la complejidad computacional.

Posteriormente, diversas investigaciones han explorado las aplicaciones prácticas y mejoras del algoritmo original. Como señalan \citet{Pan1984}, las implementaciones modernas del algoritmo de Strassen han demostrado su utilidad en contextos específicos donde las matrices son de gran tamaño, aunque con consideraciones importantes sobre estabilidad numérica. Estudios más recientes, como el de \citet{Demmel2007}, han abordado estas limitaciones mediante técnicas de precisión mixta, expandiendo así el rango de aplicaciones donde el algoritmo resulta ventajoso desde el punto de vista computacional.
\chapter{Metodología}

\section{Descripción del algoritmo}


\textit{CODIGO FUENTE}
\begin{verbatim}
@app.post("/items/")
def create_item(item: Item):
    """
    Crea un nuevo ítem en la API REST.
    """
    total = item.price * item.quantity
    return {
        "message": "Item creado exitosamente",
        "item": item,
        "total": total
    }
\end{verbatim}

\textit{TABLAS}
\begin{table}[H]
    \centering
    \caption{Tiempos de ejecución (ms)}
    \begin{tabular}{@{}cc@{}}
        \toprule
        Tamaño ($n$) & Tiempo \\
        \midrule
        100 & 0.5 \\
        1000 & 5.2 \\
        ... & ... \\
        \bottomrule
    \end{tabular}
\end{table}

\textit{FIGURAS - IMAGENES}


\begin{figure}[H]
\centering
\includegraphics[width=0.7\textwidth]{images/cnn.png}
\caption{Redes neuronales convolucionales}
\label{fig:cnn}
\end{figure}

\begin{figure}[H]
\centering
\includegraphics[width=0.7\textwidth]{images/ArbolRecursividadStrassen.jpg}
\caption{Comparación gráfica de los tiempos de ejecución}
\label{fig:resultados_strassen}
\end{figure}

\chapter{Resultados}

\section{Resultados}

Se implementó el algoritmo de Strassen en el lenguaje de programación Python, y se realizó una comparación de su desempeño con el método clásico de multiplicación de matrices. Para ello, se ejecutaron ambos algoritmos utilizando matrices cuadradas de diferentes dimensiones, midiendo el tiempo promedio de ejecución en segundos.

\begin{table}[H]
\centering
\caption{Comparación de tiempos de ejecución entre los métodos clásico y de Strassen}
\label{tab:comparacion_tiempos}
\begin{tabular}{ccc}
\hline
\textbf{Tamaño de la matriz} & \textbf{Método clásico (s)} & \textbf{Strassen (s)} \\ \hline
$64 \times 64$ & 0.012 & 0.015 \\
$128 \times 128$ & 0.095 & 0.072 \\
$256 \times 256$ & 0.812 & 0.543 \\
$512 \times 512$ & 6.427 & 3.801 \\
$1024 \times 1024$ & 50.372 & 27.156 \\ \hline
\end{tabular}
\end{table}


Como se observa en la Tabla \ref{tab:comparacion_tiempos}, el algoritmo de Strassen presenta una mejora significativa en los tiempos de ejecución a medida que aumenta el tamaño de las matrices. En el caso de las matrices más grandes ($1024 \times 1024$), el método de Strassen logra reducir el tiempo de procesamiento en aproximadamente un 46\% respecto al método clásico.

\begin{figure}[H]
\centering
\includegraphics[width=0.7\textwidth]{images/ArbolRecursividadStrassen.jpg}
\caption{Comparación gráfica de los tiempos de ejecución}
\label{fig:resultados_strassen}
\end{figure}


La Figura \ref{fig:resultados_strassen} evidencia que la ventaja del algoritmo de Strassen se vuelve más notoria a partir de matrices de tamaño medio en adelante, debido a la reducción en la complejidad computacional que introduce la descomposición recursiva de submatrices.

Finalmente, los resultados numéricos obtenidos con ambos métodos fueron equivalentes, con diferencias menores a $10^{-3}$ en los valores individuales, confirmando que el algoritmo de Strassen mantiene la exactitud de la multiplicación tradicional, pero con un mejor desempeño computacional en matrices de gran tamaño.

\chapter{Discusión}
\section{Comparación teórico-empírica}
[Análisis de convergencia/divergencia entre resultados].

\section{Factores de discrepancia}
[Influencia de hardware, SO, optimizaciones].

\section{Limitaciones}
[Problemas metodológicos encontrados].

\section{Propuestas de mejora}
[Sugerencias para futuros experimentos].
\chapter{Conclusiones}

El presente trabajo permitió implementar y analizar el desempeño del algoritmo de Strassen para la multiplicación de matrices, en comparación con el método clásico. Los resultados demostraron que el algoritmo de Strassen mejora significativamente los tiempos de ejecución cuando se trabaja con matrices de gran tamaño, reduciendo el tiempo de cómputo hasta en un 46\%.

Asimismo, se comprobó que la precisión numérica de los resultados obtenidos por el algoritmo de Strassen es equivalente a la del método tradicional, con errores mínimos en el orden de $10^{-3}$. Esto confirma que la optimización lograda no compromete la exactitud de los cálculos.

Entre las limitaciones observadas, se destaca que para matrices pequeñas el algoritmo de Strassen no representa una mejora notable, debido a la sobrecarga computacional generada por las operaciones recursivas.

Finalmente, se recomienda explorar la implementación del algoritmo de Strassen en entornos paralelos o distribuidos, así como su integración con bibliotecas optimizadas de álgebra lineal, lo que podría potenciar aún más su eficiencia en aplicaciones de alto rendimiento.


% Bibliografía
\printbibliography

% Anexos
\appendix
\chapter{Anexos}
\section{Código completo}
[Archivo fuente completo o referencia al repositorio].

\section{Conjuntos de datos}
[Descripción de datasets utilizados].

\section{Resultados brutos}
[Tablas completas de mediciones].

\end{document}